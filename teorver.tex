\documentclass[main.tex]{subfiles}
\begin{document}

\paragraph{Закон больших чисел}
Пусть $\{\xi_n\}_{n\in\mathbb{N}}$
\begin{theorem}[ЗБЧ Маркова]
	Пусть $\sE\xi_n=a_n,\sD\xi_n=\sigma_n^2$, кроме того, выполнено условие Маркова ${\lim_{n\rightarrow\infty}\sD(\frac{1}{n^2}(\sum_{i=1}^n)) = 0}$, тогда выполняется ЗБЧ
	$$\lim_{n\rightarrow\infty}P(|\frac{1}{n}\sum_{i=1}^n \xi_i - \frac{1}{n}\sum_{i=1}^n a_i|\geq\epsilon) = 0, \forall \epsilon$$
\end{theorem}

\begin{theorem}[ЗБЧ Чебышева]
	Пусть $\{\xi_n\}_{n\in\mathbb{N}}$ попарно не корелированы ($\text{cov}(\xi_i,\xi_j)=0,\forall i \neq j$)
	$\sD\xi_n \leq C$, тогда выполняется ЗБЧ
	$$\lim_{n\rightarrow\infty}P(|\frac{1}{n}\sum_{i=1}^n \xi_i - \frac{1}{n}\sum_{i=1}^n a_i|\geq\epsilon) = 0, \forall \epsilon$$
\end{theorem}

\begin{theorem}[ЗБЧ Хинтча]
	Пусть $\{\xi_n\}_{n\in\mathbb{N}}$ попарно независимые, одинаково распределённые, $|\sE\xi_i|<\infty$, тогда выполняется ЗБЧ
\end{theorem}

\end{document}