\documentclass[main.tex]{subfiles}
\begin{document}

% 6
\paragraph{Основные определения}
\begin{definition}[Статистический эксперимент]
	Тройка $(\sX,\sF,\sP)$ называется статистическим экспериментом
	\begin{itemize}
		\item $\sX$ - Множество результатов эксперимента
		\item $\sF$ - Совокупность наблюдаемых событий
		\item $\sP=\{P_\theta, \theta \in \Theta\}$ - Семейство вероятностных распределений
	\end{itemize}
\end{definition}
Дальше положим $\sX=\mathbb{R}^n$, $\sF=\sigma(\sF_1 \times \dots \times \sF_n)=\mathfrak{B}_n$

% 8
\begin{definition}[Статистика]
	Измеримая функция $T:\sX \rightarrow E$ называется статистикой
\end{definition}

\begin{definition}[Подчиненная статистика]
	Статистика $T$ называется подчиненной, если её распределение не зависит от параметра
	$$P_\theta (T\in A) = P_T(A)$$
\end{definition}

\begin{definition}[Достаточная статистика]
	Статистика $T$ назвается достаточной, если условное распределение $X$ при условии $T$ не зависит от параметра
	$$P_\theta(X\in A|T) = P_{X|T}(A),\forall \theta \in \Theta $$
\end{definition}

Подчиненная не содержит информации о параметре, достаточная содержит всю информацию о параметре

\begin{definition}[Минимальная достаточная статистика]
	Достаточная статистика $T$ называется минимальной, если, $\forall T_1$ достаточной ${\exists g : T=g(T_1)}$
\end{definition}
Использование МДС максимально редуцирует имеющиеся данные

\paragraph{Основные типы задач статистики}
\begin{itemize}
	\item Точечное оценивание (статистики $\delta : \sX \rightarrow \Theta$)
	\item Доверительное оценивание с уровнем доверия $1-\alpha$ ($\mathcal{Y}$ - семейство подмножеств $\Theta$)
		$$\Delta : \sX \rightarrow \mathcal{Y}$$
	такие, что $P_\theta (\theta \in \Delta (\vec{X})) \geq 1-\alpha, \forall \theta \in \Theta$
	\item Проверка гипотез (принятие решений) \\
	$H:\theta \in \Theta_*, \Theta_* \subset \Theta$ - Гипотеза. Выдвигают $H_0 : \theta \in \Theta_0$ и $H_A : \theta \in \Theta$ Решающее правило - критерий
	$$\phi : \sX \rightarrow [0;1]$$
	$\phi(\vec{X})$ - вероятность выбрать альтернативу (отвергнуть $H_0$)
\end{itemize}

\paragraph{Асимптотический подход}
Пусть $(\sX^{(n)},\sF^{(n)},\sP^{(n)})$ последовательность статистических экспериментов
$\sP^{(n)}=\{p_\theta^{(n)},\theta \in \Theta\}$

\begin{definition}[Состоятельность оценки]
	 Точечная оценка $\delta^{(n)}(\vec{X})$ называется состоятельной, если
	 $$\delta^{(n)}(\vec{X}) \xrightarrow{p_\theta} \theta,\forall \theta \in \Theta $$
\end{definition}
 
\begin{definition}[Сильная состоятельность оценки]
	Точечная оценка $\delta^{(n)}(\vec{X})$ называется сильно состоятельной, если
	 $$\delta^{(n)}(\vec{X}) \xrightarrow[n \rightarrow \infty]{p_\theta = 1} \theta,\forall \theta \in \Theta $$
\end{definition}

\begin{definition}[Асимптотическая нормальность]
	Точечная оценка $\delta^{(n)}(\vec{X})$ называется асимптотически нормальной, если
	$$\sqrt{n} (\delta^{(n)}(\vec{X}) - \theta) \underset{P_\theta}{\Rightarrow} \mathcal{N}(0,\sigma^2(\theta))$$
\end{definition}

\paragraph{Методы накопления статистической информации}
\begin{itemize}
	\item Выборочный метод
	\begin{definition}[Выборка]
		набор НОРСВ $\vec{X}=(x_1,\dots,x_n)$ наызывается выборкой
	\end{definition}
	Совместное распределение задается распрелелением $x_1$,  $\Theta$ не меняется с ростом $n$
	\item Группировка\newline
	Разбиваем наблюдения на $k$ групп. Наблюдаемые величины: $(x_1,z_1),\dots,(x_n,z_n)\in\sX$, где $x_i$ - наблюдаемое значение, а $z_i \in \{1,\dots,k\{$ - принадлежность его к какой-либо группе. Распределения в пределах группы совпадают. Совместное распределение определяется распределением при каждом $z_i$ - $F_s$, $s=1,\dots,k$
	\item Регрессия
	\begin{definition}[Регрессия]
		Регрессией величины $Y$ по $X$ называют $\sE[Y|X]=f(x)$
	\end{definition}
	Модель основана на соотношении $$\sE_\theta[Y,X(z)]=g_\theta(X(z)^T)$$
	Распределение $Y$ характеризуется $F_z$

\end{itemize}

\paragraph{Параметризация}
$$\sP=\{p_\theta,\theta\in\Theta\}$$
По типу множества $\Theta$ можно выделить
\begin{itemize}
	\item Параметрические $\Theta \subset \R^d$
	\item Семипараметрическими $\Theta \subset \Theta_1\times\Theta_2, \Theta_1\subset\R^d$
	\item Непараметрические (остальные)
\end{itemize}

Индикатор $\mathbbm{1}$

\end{document}
