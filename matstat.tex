\documentclass[main.tex]{subfiles}
\begin{document}

% 6
\paragraph{Основные определения}
\begin{definition}[Статистический эксперимент]
	Тройка $(\sX,\sF,\sP)$ называется статистическим экспериментом
	\begin{itemize}
		\item $\sX$ - Множество результатов эксперимента
		\item $\sF$ - Савокупность наблюдаемых событий
		\item $\sP=\{P_\theta, \theta \in \Theta\}$ - Семейство вероятностных распределений
	\end{itemize}
\end{definition}
Дальше положим $\sX=\mathbb{R}^n$, $\sF=\sigma(\sF_1 \times \dots \times \sF_n)=\mathfrak{B}_n$

% 8
\begin{definition}[Статистика]
	Измеримая функция $T:\sX \rightarrow E$ называется статистикой
\end{definition}

\begin{definition}[Подчиненная статистика]
	Статистика $T$ называется подчиненной, если её распределение не зависит от параметра
	$$P_\theta (T\in A) = P_T(A)$$
\end{definition}

\begin{definition}[Достаточная статистика]
	Статистика $T$ назвается достаточной, если условное распределение $X$ при условии $T$ не зависит от параметра
	$$P_\theta(X\in A|T) = P_{X|T}(A),\forall \theta \in \Theta $$
\end{definition}

Подчиненная не содержит информации о параметре, достаточная содержит всю информацию о параметре

\begin{definition}[Минимальная достаточная статистика]
	Достаточная статистика $T$ называется минимальной, если, $\forall T_1$ достаточной ${\exists g : T=g(T_1)}$
\end{definition}
Использование МДС максимально редуцирует имеющиеся данные

\paragraph{Основные типы задачь статистики}
\begin{itemize}
	\item Точечное оценивание (статистики $\delta : \sX \rightarrow \Theta$)
	\item Доверительное оценивание с уровнеи доверия $1-\alpha$ ($\mathcal{Y}$ - семейство подмножеств $\Theta$)
		$$\Delta : \sX \rightarrow \mathcal{Y}$$
	такие, что $P_\theta (\theta \in \Delta (\vec{X})) \geq 1-\alpha, \forall \theta \in \Theta$
	\item Проверка гипотез (принятие решений) \\
	$H:\theta \in \Theta_*, \Theta_* \subset \Theta$ - Гипотеза. Выдвигают $H_0 : \theta \in \Theta_0$ и $H_A : \theta \in \Theta$ Решающее правило - критерий
	$$\phi : \sX \rightarrow [0;1]$$
	$\phi(\vec{X})$ - вероятность выбрать альтернативу (отвергнуть $H_0$)
\end{itemize}

\paragraph{Асимптотический подход}
Пусть $(\sX^{(n)},\sF^{(n)},\sP^{(n)})$ последовательность статистичиеских экспериментов
$\sP^{(n)}=\{p_\theta^{(n)},\theta \in \Theta\}$

\begin{definition}[Состоятельность оценки]
	 Точечная оценка $\delta^{(n)}(\vec{X})$ называется состоятельной, если
	 $$\delta^{(n)}(\vec{X}) \xrightarrow{p_\theta} \theta,\forall \theta \in \Theta $$
\end{definition}

\begin{definition}[Сильная состоятельность оценки]
	Точечная оценка $\delta^{(n)}(\vec{X})$ называется сильно состоятельной, если
	 $$\delta^{(n)}(\vec{X}) \xrightarrow[n \rightarrow \infty]{p_\theta = 1} \theta,\forall \theta \in \Theta $$
\end{definition}

\begin{definition}[Асимптотическая нормальность]
	Точечьная оценка $\delta^{(n)}(\vec{X})$ называется асимптотически нормальной, если
	$$\sqrt{n} (\delta^{(n)}(\vec{X}) - \theta) \underset{P_\theta}{\Rightarrow} \mathcal{N}(0,\sigma^2(\theta))$$
\end{definition}
Индикатор $\mathbbm{1}$

\end{document}