\documentclass[main.tex]{subfiles}
\begin{document}

% 6
\begin{definition}[Статистический эксперимент]
	Тройка $(\sX,\sF,\sP)$ называется статистическим экспериментом\newline
	$\sX$ - Множество результатов эксперимента\newline
	$\sF$ - Савокупность наблюдаемых событий\newline
	$\sP=\{P_\theta, \theta \in \Theta\}$ - Семейство вероятностных распределений
\end{definition}
Дальше положим $\sX=\mathbb{R}^n$, $\sF=\sigma(\sF_1 \times \dots \times \sF_n)=\mathfrak{B}_n$

% 8
\begin{definition}[Статистика]
	Измеримая функция $T:\sX \rightarrow E$ называется статистикой
\end{definition}

\begin{definition}[Подчиненная статистика]
	Статистика $T$ называется подчиненной, если её распределение не зависит от параметра
	$$P_\theta (T\in A) = P_T(A)$$
\end{definition}

\begin{definition}[Достаточная статистика]
	Статистика $T$ назвается достаточной, если условное распределение $X$ при условии $T$ не зависит от параметра
	$$P_\theta(X\in A|T) = P_{X|T}(A),\forall \theta \in \Theta $$
\end{definition}

Подчиненная не содержит информации о параметре, достаточная содержит всю информацию о параметре

\begin{definition}[Минимальная достаточная статистика]
	Достаточная статистика $T$ называется минимальной, если, $\forall T_1$ достаточной ${\exists g : T=g(T_1)}$
\end{definition}

\end{document}