\documentclass{article}
\usepackage[T2A]{fontenc}
\usepackage[utf8x]{inputenc}
\usepackage[english, russian]{babel} % Localisation

% \usepackage{fontspec} % loaded by polyglossia, but included here for transparency 
% \usepackage{polyglossia}
% \setmainlanguage{russian} 
% \setotherlanguage{english}

\usepackage{amssymb} % More math symbols
\usepackage{amsmath} % Math constructions
\usepackage{amsthm} % Theorems
\usepackage{subfiles} % File separation

\newcommand{\sP}{\mathcal{P}}
\newcommand{\sF}{\mathfrak{F}}
\newcommand{\sX}{\mathfrak{X}}

\newtheorem{lemma}{Лемма}
\newtheorem{definition}{Определение}

\begin{document}

\title{Этакое большое ничего и матстат}
\author{Белкин Дмитрий, студент группы 4362
\and Бертыш Вадим, студент группы 4373}
\date{15 июня 2016}
\maketitle
\newpage

% 6
\begin{definition}[Статистический эксперимент]
	Тройка $(\sX,\sF,\sP)$ называется статистическим экспериментом\newline
	$\sX$ - Множество результатов эксперимента\newline
	$\sF$ - Савокупность наблюдаемых событий\newline
	$\sP=\{P_\theta, \theta \in \Theta\}$ - Семейство вероятностных распределений
\end{definition}
Дальше положим $\sX=\mathbb{R}^n$, $\sF=\sigma(\sF_1 \times \dots \times \sF_n)$

% 8
\begin{definition}[Статистика]
	Измеримая функция $T:\sX \rightarrow E$ называется статистикой
\end{definition}

\begin{definition}[Подчиненная статистика]
	Статистика $T$ называется подчиненной, если её распределение не зависит от параметра
	$$P_\theta (T\in A) = P_T(A)$$
\end{definition}

\begin{definition}[Достаточная статистика]
	Статистика $T$ назвается достаточной, если условное распределение $X$ при условии $T$ не зависит от параметра
	$$P_\theta(X\in A|T) = P_{X|T}(A),\forall \theta \in \Theta $$
\end{definition}


\end{document}