\documentclass[main.tex]{subfiles}
\begin{document}

\paragraph{Регрессионный анализ}
% 38
\begin{definition}[Регрессия] ~\\
	Пусть $Y$ - наблюдение,  $Z$ - характеристика, определяющая распределение $Y$, $F_Z$ - распределение $Y$ при фиксированном $Z$.\\
	Пусть $Y_1,\dots,Y_n$ - независимы.
	Установим зависимость $Y_i$ от $i$.
	Сопоставим $\forall i: i \mapsto Z_i \implies F_i \equiv F_{Z_i}$.
	Обычно эту зависимость задают параметрически ($\ex: F_i = g_\theta(F_{Z_0}), \theta \in \R^d$).\\
	Тогда $E_\theta(Y|Z) = g_\theta(Z)$ - \textbf{регрессия} $Y$ по $Z$.
\end{definition}

\begin{definition}[Линейная регрессия] ~\\
	Регрессия называется \textbf{линейной} если
	\[\exists X(Z)=
		\begin{pmatrix}
		X_1(Z)\\
		\vdots\\
		X_n(Z)
		\end{pmatrix}
		\text{ - регрессор.} 
	\]
	\textbf{Модель линейной регрессии}
	\[\sE_\theta(Y|Z) = X^T\beta,\ \beta =
		\begin{pmatrix}
			\beta_1\\
			\vdots\\
			\beta_n
		\end{pmatrix}
	\]
	В условиях этой модели
	\begin{align*}
		\sE_\theta Y_i &= (X(Z_i))^T\beta\\
		Y_i &= (X(Z_i))^T\beta + \varepsilon_i\\
		Y &= X^T\beta + \varepsilon (\sE\varepsilon = 0)\\
	\end{align*}
	Где $X \in M_{m\times n}$ - матрица регрессоров, $\beta$ - $m \times 1$ - столбец параметров, $\varepsilon = (\varepsilon_1,\dots,\varepsilon_n)^T\text{ - вектор отклонений.}$
\end{definition}
\textbf{Примеры регрессионных моделей.}
$Y_1,\dots,Y_n$ - независимые наблюдения.
\begin{enumerate}
	\item Выборка
	\[\sE Y_i = \beta_i \text{ Если $\varepsilon_i$ - НОРСВ, то } Y_1,\dots,Y_n\]
	\item Простая регрессионная модель
	\[Y_i = \beta_1 + \beta_2 Z_i + \varepsilon_i,\ X = \begin{pmatrix}
		1   & \cdots & 1\\
		Z_1 & \cdots & Z_n\\
	\end{pmatrix}\]
	\item Полиномиальная модель
	\[Y_i = \sum_{j=1}^s \beta_jZ_i^{j-1},\ X = \begin{pmatrix}
		1 & \cdots & 1\\
		Z_1 & \cdots & Z_n\\
		Z_1^2 & \cdots & Z_n^2\\
		\vdots & \ddots & \vdots\\
		Z_1^{s-1} & \cdots & Z_n^{s-1}
	\end{pmatrix}\]
	\item Простая группировка (однофакторный дисперсионный анализ)
	\begin{align*}
		&Z \in \{1,\dots,I\}\\
		&\beta = (\beta_1,\dots,\beta_I)^T\\
		&\sE(Y|Z) = \beta_Z\\
		&Y_i = \beta_{Z_i} + \varepsilon_i\\
		&\forall i < j: Z_i <= Z_j\\
		&X = \begin{pmatrix}
			1 & 1 & 0 & 0 & & & &\\
			0 & 0 & 1 & 1 & &  \textbf{0} & &\\
			0 & 0 & 0 & 0 & & & &\\
			& & & & \ddots & & &\\
			& & \textbf{0} & & & \ddots & &\\
			& & & & & & 1 & 1
		\end{pmatrix}
	\end{align*}
\end{enumerate}
\end{document}
