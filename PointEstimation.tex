\documentclass[main.tex]{subfiles}
\begin{document}

\paragraph{Непараметрическое оценивание (Выборочный подход)}
Пусть $X_1,\dots\,X_n$ - выборка из распределения $P_\theta, \theta\in\Theta$, истиное значение $P_\theta$ будем называеть {\itтеоретическим распределением}.
\begin{definition}[Выборочная функция распределения]
	Выборочной функцией распределения называют $$F_n(x)=\frac{1}{n}\sum_{i=1}^n\mathbbm{1}_{\{X_i < x\}}=\frac{\text{Число наблюдений меньших }x}{\text{общее число наблюдений}}$$
\end{definition}

\begin{theorem}[Гливенко-Кантелли]
	$$\sup_x \{|F_n(x)-F(x)|\} \xrightarrow[n\rightarrow \infty]{P_\theta=1} 0,\forall F$$
\end{theorem}
Утверждает сильную состочтельность $F_n$

\begin{theorem}[Колмогорова]
	$$\sqrt{n} \sup_x \{F_n(x)-F(x)\} \underset{F}{\Rightarrow}\mathcal{K}$$
\end{theorem}
Где $\mathcal{K}$ -  распределение Колмогорова, функция распределения ${\mathcal{K}(x) = \sum_{j=-\infty}^{\infty} (-1)^j e^{-2j^2x^2}}$

\paragraph{Параметрическое оценивание}
\begin{definition}[Точечная оценка]
	Статистика $\delta ( \vec{X} )$, $\delta : \sX \rightarrow \Theta$ называется точечной оценкой
\end{definition}

\begin{definition}[Функция потерь]
	пусть $\theta$ реально значение параметра, тогда $W(\delta (\vec{X}),\theta)$ функция потерь, если
	\begin{itemize}
	 	\item $W(\delta (\vec{X}),\theta)>0,\forall \vec{X} \in \sX$
	 	\item $W(\theta,\theta)=0$
	 \end{itemize} 
\end{definition}
Используют различные функции потерь (в дальнейшем используем функцию Гаусса)
\begin{align}
	& W(\delta,\theta)=|\delta-\theta| \tag{Лаплас} \\
	& W(\delta,\theta)=(\delta-\theta)^2 \tag{Гаусс}
\end{align}

\begin{definition}[Риск]
	Риском называют $R(\delta,\theta) = \sE_\theta[W(\delta(\vec{X}),\theta)]$
\end{definition}

\end{document}
