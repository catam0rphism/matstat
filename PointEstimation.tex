\documentclass[main.tex]{subfiles}
\begin{document}

\paragraph{Непараметрическое оценивание (Выборочный подход)}
Пусть $X_1,\dots\,X_n$ - выборка из распределения $P_\theta, \theta\in\Theta$, истиное значение $P_\theta$ будем называеть {\itтеоретическим распределением}.
\begin{definition}[Выборочная функция распределения]
	Выборочной функцией распределения называют $$F_n(x)=\frac{1}{n}\sum_{i=1}^n\mathbbm{1}_{\{X_i < x\}}=\frac{\text{Число наблюдений меньших }x}{\text{общее число наблюдений}}$$
\end{definition}

\begin{theorem}[Гливенко-Кантелли]
	$$\sup_x \{|F_n(x)-F(x)|\} \xrightarrow[n\rightarrow \infty]{P_\theta=1} 0,\forall F$$
\end{theorem}
Утверждает сильную состочтельность $F_n$

\begin{theorem}[Колмогорова]
	$$\sqrt{n} \sup_x \{F_n(x)-F(x)\} \underset{F}{\Rightarrow}\mathcal{K}$$
\end{theorem}
Где $\mathcal{K}$ -  распределение Колмогорова, функция распределения ${\mathcal{K}(x) = \sum_{j=-\infty}^{\infty} (-1)^j e^{-2j^2x^2}}$
Даёт возможность строить доверительные области для функции распределения. Пусть $\alpha$ - малое число, $x_\alpha$ - квантиль уровня $1-\alpha$, тогда
$$1-\alpha \approx P_F(\sqrt(n) \sup_x |F_n(x)-F(x)|<\alpha)={P_F(F_n(x)-\frac{x_\alpha}{\sqrt{n}} \leq F(x) \leq F_n(x)+\frac{x_\alpha}{\sqrt{n}})}$$
Идея доказательства - преобразование Смирнова
\begin{definition}[Преобразование Смирнова]
	$x$ с непренывной функцией распределения $F$, $Y=F(x)$, тогда $$Y\sim \mathcal{U}(0,1)$$ 
\end{definition}

\begin{definition}[Порядковые статистики, вариационный ряд и ранги] Достаточная статистика
	$$X_{(1)}\leq X_{(2)}\leq \dots X_{(n)}$$ называется вариационным рядом, $X_(k)$ - $k$-ая порядковая статистика. Ранг $R_k$ - номер $x_k$ в вариационном ряду
\end{definition}

\paragraph{Выборочные характеристики}
\begin{definition}[Выборочные характеристики]
	Пусть $\mathscr{F}$ - подмножество множество распределений $\alpha:\mathscr{F}\rightarrow\R$ - числовая характеристика. Тогда 
	\begin{align*}
		&\alpha(F) \text{ - теоретическая характеристика}\\
		&\alpha(F_n) \text{ - выборочная характеристика}\\
	\end{align*}
\end{definition}

Числовые характеристики можно разделить на две вида:
\begin{itemize}
	\item Характеристики $H(\sE g_1(X), \dots, \sE g_n(X))$
	\item Непрерывный в равномерной метрике функционал $G(F)$
\end{itemize}
К первым относятся моментные характеристики, ко вторым квантили.
\begin{theorem}
	Пусть $X_1,\dots,X_n$ - Выборка из распределения с функцией распределения $F$, числовая характеристика $G(F)$ первого или второго типа существует, тогда с вероятностью 1
	$$G(F_n) \underset{n\rightarrow\infty}{\rightarrow}G(F)$$
\end{theorem}

Выборочные квантили
% -------------------------------------------------------
\paragraph{Параметрическое оценивание}
\begin{definition}[Точечная оценка]
	Статистика $\delta ( \vec{X} )$, $\delta : \sX \rightarrow \Theta$ называется точечной оценкой
\end{definition}

\begin{definition}[Функция потерь]
	пусть $\theta$ реально значение параметра, тогда $W(\delta (\vec{X}),\theta)$ функция потерь, если
	\begin{itemize}
	 	\item $W(\delta (\vec{X}),\theta)>0,\forall \vec{X} \in \sX$
	 	\item $W(\theta,\theta)=0$
	 \end{itemize} 
\end{definition}
Используют различные функции потерь (в дальнейшем используем функцию Гаусса)
\begin{align}
	& W(\delta,\theta)=|\delta-\theta| \tag{Лаплас} \\
	& W(\delta,\theta)=(\delta-\theta)^2 \tag{Гаусс}
\end{align}

\begin{definition}[Риск]
	Риском называют $R(\delta,\theta) = \sE_\theta[W(\delta(\vec{X}),\theta)]$
\end{definition}

\end{document}
