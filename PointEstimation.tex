\documentclass[main.tex]{subfiles}
\begin{document}

\paragraph{Непараметрическое оценивание (Выборочный подход)}
Пусть $X_1,\dots\,X_n$ - выборка из распределения $P_\theta, \theta\in\Theta$, истиное значение $P_\theta$ будем называеть {\itтеоретическим распределением}.
\begin{definition}[Выборочная функция распределения]
	Выборочной функцией распределения называют $$F_n(x)=\frac{1}{n}\sum_{i=1}^n\mathbbm{1}_{\{X_i < x\}}=\frac{\text{Число наблюдений меньших }x}{\text{общее число наблюдений}}$$
\end{definition}

\begin{theorem}[Гливенко-Кантелли]
	$$\sup_x \{|F_n(x)-F(x)|\} \xrightarrow[n\rightarrow \infty]{P_\theta=1} 0,\forall F$$
\end{theorem}
Утверждает сильную состоятельность $F_n$

\begin{theorem}[Колмогорова]
	$$\sqrt{n} \sup_x \{F_n(x)-F(x)\} \underset{F}{\Rightarrow}\mathcal{K}$$
\end{theorem}
Где $\mathcal{K}$ -  распределение Колмогорова, функция распределения ${\mathcal{K}(x) = \sum_{j=-\infty}^{\infty} (-1)^j e^{-2j^2x^2}}$
Даёт возможность строить доверительные области для функции распределения. Пусть $\alpha$ - малое число, $x_\alpha$ - квантиль уровня $1-\alpha$, тогда
$$1-\alpha \approx P_F(\sqrt(n) \sup_x |F_n(x)-F(x)|<\alpha)={P_F(F_n(x)-\frac{x_\alpha}{\sqrt{n}} \leq F(x) \leq F_n(x)+\frac{x_\alpha}{\sqrt{n}})}$$
Идея доказательства - преобразование Смирнова
\begin{definition}[Преобразование Смирнова]
	$x$ с непренывной функцией распределения $F$, $Y=F(x)$, тогда $$Y\sim \mathcal{U}(0,1)$$ 
\end{definition}

\begin{definition}[Порядковые статистики, вариационный ряд и ранги] Достаточная статистика
	$$X_{(1)}\leq X_{(2)}\leq \dots X_{(n)}$$ называется вариационным рядом, $X_(k)$ - $k$-ая порядковая статистика. Ранг $R_k$ - номер $x_k$ в вариационном ряду
\end{definition}

\paragraph{Выборочные характеристики}
\begin{definition}[Выборочные характеристики]
	Пусть $\mathscr{F}$ - подмножество множество распределений $\alpha:\mathscr{F}\rightarrow\R$ - числовая характеристика. Тогда 
	\begin{align*}
		&\alpha(F) \text{ - теоретическая характеристика}\\
		&\alpha(F_n) \text{ - выборочная характеристика}\\
	\end{align*}
\end{definition}

Числовые характеристики можно разделить на две вида:
\begin{itemize}
	\item Характеристики $H(\sE g_1(X), \dots, \sE g_n(X))$
	\item Непрерывный в равномерной метрике функционал $G(F)$
\end{itemize}
К первым относятся моментные характеристики, ко вторым квантили.
\begin{theorem}
	Пусть $X_1,\dots,X_n$ - Выборка из распределения с функцией распределения $F$, числовая характеристика $G(F)$ первого или второго типа существует, тогда с вероятностью 1
	$$G(F_n) \underset{n\rightarrow\infty}{\rightarrow}G(F)$$
\end{theorem}

\begin{definition}[Выборочные квантили]
	$\hat{\xi_p}$ - квантиль порядка $p$, $\hat{\xi_{p}}=X_{([np]+1)}$, если $np\notin\mathbb{Z}$, иначе $\hat{\xi_{p}}\in [X_{(np)};X_{(np+1)})$
\end{definition}

\begin{theorem}[Асимптотическая нормальность выборочных квантилей]
	Пусть $f(\xi_p)>0$, $Z_{n,p}$ - выборочная квантиль, то $$Q=\sqrt{\frac{n}{p(1-p)}}f(\xi_p)(Z_{n,p} - \xi_p) \Rightarrow \mathcal{N}(0,1)$$ то есть $$ P_F(\sqrt{\frac{n}{p(1-p)}}f(\xi_p)(Z_{n,p} - \xi_p) < x) \underset{n\rightarrow\infty}{\rightarrow} \Phi(x), \forall x$$
	где $\Phi(x) = \int_{-\infty}^{x}\frac{1}{\sqrt{2\pi}}e^{-\frac{t^2}{2}}\mathrm{d}t$ - функция распределения нормального распределения
\end{theorem}

% -------------------------------------------------------
\paragraph{Параметрическое оценивание}
\begin{definition}[Точечная оценка]
	Статистика $\delta ( \vec{X} )$, $\delta : \sX \rightarrow \Theta$ называется точечной оценкой
\end{definition}

\begin{definition}[Функция потерь]
	пусть $\theta$ реально значение параметра, тогда $W(\delta (\vec{X}),\theta)$ функция потерь, если
	\begin{itemize}
	 	\item $W(\delta (\vec{X}),\theta)>0,\forall \vec{X} \in \sX$
	 	\item $W(\theta,\theta)=0$
	 \end{itemize} 
\end{definition}
Используют различные функции потерь (в дальнейшем используем функцию Гаусса)
\begin{align}
	& W(\delta,\theta)=|\delta-\theta| \tag{Лаплас} \\
	& W(\delta,\theta)=(\delta-\theta)^2 \tag{Гаусс}
\end{align}

\begin{definition}[Риск]
	Риском называют $R(\delta,\theta) = \sE_\theta[W(\delta(\vec{X}),\theta)]$
\end{definition}

\begin{definition}[Смещение оценки]
	Смещением оценки $\delta(\vec{X})$ параметрической функцией $g(\theta)$ называют величину $$b_\theta(x)=\sE_\theta[\delta(\vec{X})]-g(\theta)$$
\end{definition}

\begin{definition}[НРМД оценка]
	Пусть $\delta_0(\vec X)$ - несмещённая оценка $g(\theta)$. Мы будем называть её несмещённой с равномерно минимальной дисперсией, если:
	\begin{gather*}
		\forall \delta(\vec X) \text{ - несмещенных оценок } g(\theta):
		\ \sD\delta_0(\vec X) \leq \sD\delta(\vec X),\ \forall \theta \in \Theta
	\end{gather*}
\end{definition}

\begin{theorem}[Единственность НРМД-оценки]
	Если $\delta_1(\vec{X})$ и $\delta_2(\vec{X})$ НРМД, то $P(\delta_1(\vec{X}) = \delta_2(\vec{X})) = 1$
\end{theorem}

\begin{theorem}[факторизация Неймана-Фишера]
	Статистика $T(\vec{X})$ достаточно тогда и только тогда, когда допустимо представление
	$$L(\vec{X};\theta) = g(T(\vec{X}),\theta)h(\vec{X})$$ где $g(t,\theta)$ и $h$ некоторые неотрицательные функции
\end{theorem}

\paragraph{Оценка максимального правдоподобия}
\begin{definition}[Функция правдоподобия] ~\\
	Пусть $X_1,\dots,X_n$ - набор независимых наблюдений с плотностями $f_{\theta,1},\dots,f_{\theta,n}$ соответственно. Мы будем называть функцией правдоподобия
	\[L(\vec X, \theta) = \prod_{i=1}^nf_{\theta,i}(X_i),\ \theta \in \Theta\]
	А значение этой функции при фиксированном $\theta$ правдоподобием наблюдений. Также используют $LL(\vec{X},\theta)=\log L(\vec{X},\theta)$
\end{definition}

\begin{definition}[Метод максимального правдоподобия]
Идея: найти значение $\hat\theta(\vec X) \in \Theta$ такое, что
	\[\forall \theta \in \Theta: L(\vec X, \hat\theta(\vec X)) \leq L(\vec X, \theta)\]
\end{definition}
	
\begin{definition}[Достаточная статистика]
	Статистика $T$ назвается достаточной, если условное распределение $X$ при условии $T$ не зависит от параметра
	$$P_\theta(X\in A|T) = P_{X|T}(A),\forall \theta \in \Theta $$
\end{definition}

\begin{theorem}[Рао -- Блэкуэлла -- Колмогорова]
	Если $T$ - достаточная статистика $\delta$ - оценка параметра $\theta$, то оценка $\sE(\delta|T)$, если $W$ выпукла (вниз), то $R(\delta,\theta)\geq R(\sE(\delta|T),\theta)$
\end{theorem}
Суть: лучшая оценка всегда может быть получена как функция от достаточных статистик.

\begin{definition}[Полная статистика]
	Статистика $T$ называется полной, если с вероятностью 1 $$\forall\theta\in\Theta.\sE_\theta g(T)=0 \rightarrow g(T)=0$$
\end{definition}
Полная достаточная статистика всегда является минимальной.
% !!!!!!!!!!!!!!!!!!!!   ???   !!!!!!!!!!!!!!!!!!!!!!!!
\begin{theorem}[Лемана-Шеффе]
	В классе оценок с данным смещением, существует не более одной, являющейся функцией от достаточной статистики
\end{theorem}
Если МДС полная, то при фиксированном значении смещения существует единственная оценка, минимизирующая риск в классе оценок с таким смещением.

\paragraph{Минимаксный и байесовский подходы}
\begin{definition}[Минимаксная оценка]
	Оценка $\delta_0$ называется минимаксной, если они минимизиреут максимальный риск $$R(\delta_0,\theta)=\inf_\delta \sup_{\theta\in\Theta} R(\delta,\theta)$$
\end{definition}

Пусть на $\Theta$ задано априорное распределение $Q$
\begin{definition}[Байесовский риск и байесовская оценка]
	Величина $$R(\delta)=\int_\Theta R(\delta,\theta)Q(\mathrm{d}\theta) $$ называется байесовским риском оценки $\delta$. Оценка, минимизирующая байесовский риск называется байесовской
\end{definition}

\begin{theorem}[Лемана]
	Пусть $\{\delta_n\}_{n\in\mathbb{N}}$ последовательность байесовских оценок по отношению к априорным распределениям $\{Q_n\}$ соответственно, тогда $\delta$ - минимаксная
	$$\delta : \sup_\theta R(\delta,\theta) \leq \underset{n\rightarrow\infty}{\lim\sup}\int_\Theta R(\delta_n,\theta)\mathrm{d}Q_n$$
\end{theorem}

\end{document}
