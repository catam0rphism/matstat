\documentclass[main.tex]{subfiles}
\begin{document}

\begin{definition}[Точечная оценка]
	Статистика $\delta ( \vec{X} )$, $\delta : \sX \rightarrow \Theta$ называется точечной оценкой
\end{definition}

\begin{definition}[Функция потерь]
	пусть $\theta$ реально значение параметра, тогда $W(\delta (\vec{X}),\theta)$ функция потерь, если
	\begin{itemize}
	 	\item $W(\delta (\vec{X}),\theta)>0,\forall \vec{X} \in \sX$
	 	\item $W(\theta,\theta)=0$
	 \end{itemize} 
\end{definition}
Используют различные функции потерь (в дальнейшем используем функцию Гаусса)
\begin{align}
	& W(\delta,\theta)=|\delta-\theta| \tag{Лаплас} \\
	& W(\delta,\theta)=(\delta-\theta)^2 \tag{Гаусс}
\end{align}

\begin{definition}[Риск]
	Риском называют $R(\delta,\theta) = \sE_\theta[W(\delta(\vec{X}),\theta)]$ 
\end{definition}

\end{document}
